%% LaTeX2e class for seminar theses
%% sections/content.tex
%% 
%% Karlsruhe Institute of Technology
%% Institute for Program Structures and Data Organization
%% Chair for Software Design and Quality (SDQ)
%%
%% Dr.-Ing. Erik Burger
%% burger@kit.edu
%%
%% Version 1.0.2, 2020-05-07

\section{Introduction}
\label{ch:Introduction}

%% -------------------
%% | Example content |
%% -------------------
The goal of depth estimation is to obtain a representation of the spatial structure of a scene, recovering the three-dimensional shape and appearance of objects in imagery. This is also known as the inverse problem [3], where we seek to recover some unknowns given insufficient information to fully specify the solution. Meaning that the mapping between the 2D view and 3D is not unique (fig 12) I will compare the performane of the classical stereo methods and deep learning methods in this thesis.

\url{https://sdqweb.ipd.kit.edu/wiki/Ausarbeitungshinweise} or to your advisor.

\subsection{Example: Citation}
\label{sec:Introduction:Citation}
A citation: \cite{becker2008a} For referencing, see \autoref{sec:Introduction:Figures}

\subsection{Example: Figures}
\label{sec:Introduction:Figures}
\begin{figure}
\centering
\includegraphics[width=4cm]{images/sdqlogo}
\caption{SDQ logo}
\label{fig:sdqlogo}
\end{figure}

A reference: The SDQ logo is displayed in \autoref{fig:sdqlogo}. 
(Use \code{\textbackslash autoref\{\}} for easy referencing.) 

\subsection{Example: Tables}
\label{sec:Introduction:Tables}
\begin{table}
\centering
\begin{tabular}{r l}
\toprule
abc & def\\
ghi & jkl\\
\midrule
123 & 456\\
789 & 0AB\\
\bottomrule
\end{tabular}
\caption{A table}
\label{tab:atable}
\end{table}

\subsection{Example: Todo-Note}
Meaningless text.

\subsection{Example: Formula}
One of the nice things about the Linux Libertine font is that it comes with
a math mode package.
\begin{displaymath}
f(x)=\Omega(g(x))\ (x\rightarrow\infty)\;\Leftrightarrow\;
\limsup_{x \to \infty} \left|\frac{f(x)}{g(x)}\right|> 0
\end{displaymath}

%% --------------------
%% | /Example content |
%% --------------------